\documentclass{article}
\usepackage{tcolorbox}
\usepackage{amsmath}
\usepackage{mathtools}
\begin{document}
\date{June 18, 2018}
\section{Data Movement Between Memory}
When a program is running inside a program, only portion of that program is being ran inside the main memory.

Temporal Locality Items accessed recently are likely to be accessed again soon.

Spatial locality is when the items near those accessed recentlyare likely to be accessed soon.

Caches generally hold the addresses frequently accessed, so, they could be pulled again.

\section{Cache Memory}
Cache memories are divided into blocks.

\subsection{}
Where each of those blocks are segregated via keys, where those keys maps to the addresses in the Main Memory.
\begin{equation}
Cache Block Number = Remainder\frac{Main Memory Block Number}{Number of blocks in cache}
\end{equation}

\end{document}
